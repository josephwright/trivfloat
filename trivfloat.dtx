% \iffalse meta-comment
%
% Copyright (C) 2007 by
%    Joseph Wright <joseph.wright@morningstar2.co.uk>
%
% Released under the GNU General Public License v2
% See http://www.gnu.org/licenses/gpl.txt
%
% This work consists of the main source file chemextra.dtx
% and the derived files
%
%<*driver>
\documentclass{ltxdoc}
\EnableCrossrefs
\CodelineIndex
\RecordChanges
%\OnlyDescription
\usepackage[T1]{fontenc}
\usepackage[english,UKenglish]{babel}
\usepackage{mathpazo,booktabs,array,url}
\usepackage{\jobname}
\begin{document}
  \DocInput{\jobname.dtx}
\end{document}
%</driver>
% \fi
%
% \CheckSum{107}
%
% \CharacterTable
%  {Upper-case    \A\B\C\D\E\F\G\H\I\J\K\L\M\N\O\P\Q\R\S\T\U\V\W\X\Y\Z
%   Lower-case    \a\b\c\d\e\f\g\h\i\j\k\l\m\n\o\p\q\r\s\t\u\v\w\x\y\z
%   Digits        \0\1\2\3\4\5\6\7\8\9
%   Exclamation   \!     Double quote  \"     Hash (number) \#
%   Dollar        \$     Percent       \%     Ampersand     \&
%   Acute accent  \'     Left paren    \(     Right paren   \)
%   Asterisk      \*     Plus          \+     Comma         \,
%   Minus         \-     Point         \.     Solidus       \/
%   Colon         \:     Semicolon     \;     Less than     \<
%   Equals        \=     Greater than  \>     Question mark \?
%   Commercial at \@     Left bracket  \[     Backslash     \\
%   Right bracket \]     Circumflex    \^     Underscore    \_
%   Grave accent  \`     Left brace    \{     Vertical bar  \|
%   Right brace   \}     Tilde         \~}
%
% \GetFileInfo{\jobname.sty}
%
% \changes{v1.0}{2007/06/09}{Initial public release}
%
% \DoNotIndex{\@eha,\@ne,\advance,\begingroup,\csname,\def,\edef}
% \DoNotIndex{\else,\end,\endcsname,\endgroup,\expandafter,\fi}
% \DoNotIndex{\ifcase,\jobname,\MakeLowercase,\MakeUppercase}
% \DoNotIndex{\MessageBreak,\NeedsTeXFormat,\newcommand,\newcount}
% \DoNotIndex{\noexpand,\or,\PackageError,\PackageInfo}
% \DoNotIndex{\PackageWarning,\ProvidesPackage,\RequirePackage}
% \DoNotIndex{\romannumeral,\x}
%
% \title{Quick floats in LaTeX}
% \author{Joseph Wright%
% \thanks{E-mail: joseph.wright@morningstar2.co.uk}}
% \date{\filedate}
%
% \maketitle
%
% \begin{abstract}
%
% The |trivfloat| package provides a quick method for defining new
% float types in LaTeX.  A single command sets up a new float in the
% same style as the LaTeX kernel |figure| and |table| float types.
%
% \end{abstract}
%
% \section{Introduction}
%
% The LaTeX kernel provides the |figure| and |table| floating
% environment, but does not provide an easy method for defining new
% float types.  This problem is addressed by the |float| package,
% which provides an array of commands to create new float types.
% However, the one command missing is a quick-and-dirty one to
% set up a new float type with no customisation.  This is addressed
% by the |trivfloat| package.
%
% \section{Usage}
%
% \DescribeMacro{\trivfloat}
% \DescribeMacro{\listoffloat-names}
% The package provides a single command to produce the new float
% type, |\trvifloat|.  This takes a single argument
% \marg{float-name}, which will be the name of the new floating
% environment.  The new environment can be used as normal; the new
% floats should behave exactly the same as |figure| and |table|
% environments.  The command |\listof|\meta{float-name}|s| is also
% provided.
%
% \StopEventually{}
%
% \iffalse
%<*package>
% \fi
%
% \section{Implementation}
%
% \subsection{Setup code}
%
% The first part of the package is concerned with basic
% identification and loading support packages.
%    \begin{macrocode}
\NeedsTeXFormat{LaTeX2e}
\ProvidesPackage{trivfloat}
  [2007/06/09 v1.0 Quick floats in LaTeX]
\RequirePackage{float}
%    \end{macrocode}
%
% \subsection{Internal macros}
%
% \begin{macro}{\tfl@floatcount}
% A new counter is needed to track how many floats have been
% generated by |trivfloat|.
%    \begin{macrocode}
\newcount\tfl@floatcount%
%    \end{macrocode}
% \end{macro}
% \begin{macro}{\tfl@genext}
% \begin{macro}{\tfl@ext@i}
% Each float type needs a list to store entries when generating
% a contents listing.  Rather than try complex things with the
% initial characters of the float name, an arbitrary file extension
% is used.  Note that |lof|, |log| and |lot| are taken by default,
% so we don't use them.  Also notice that if people try to make too
% many new float types, then extension |xxx| will be used.  This is
% very unlikely, but odd things might happen so we warn them.
%    \begin{macrocode}
\def\tfl@genext{%
  \expandafter\def\csname tfl@ext@\romannumeral\tfl@floatcount%
      \expandafter\endcsname{%
    \ifcase\tfl@floatcount%
      \PackageError{trivfloat}%
        {Something's wrong here --- Contact the package author!}%
      \@eha%
    \or % 1
      loa%
    \or % 2
      lob%
    \or % 3
      loc%
    \or % 4
      lod%
    \or % 5
      loe%
    \or % 6
      loh%
    \or % 7
      loi%
    \or % 8
      loj%
    \or % 9
      lok%
    \or % 10
      lol%
    \or % 11
      lom%
    \or % 12
      lon%
    \or % 13
      loo%
    \or % 15
      loq%
    \or % 16
      lor%
    \or % 17
      los%
    \or % 18
      lou%
    \or % 19
      lov%
    \or % 20
      low%
    \or % 21
      lox%
    \or % 22
      loy%
    \or % 23
      loz%
    \else % 24 or more
      \PackageWarning{trivfloat}%
        {I've run out of extensions \MessageBreak%
         I'm using \jobname.xxx to list all future floats}%
      xxx%
    \fi%
  }%
}%
%    \end{macrocode}
% \end{macro}
% \end{macro}
% \begin{macro}{\tfl@gennamea}
% \begin{macro}{\tfl@genname}
% \begin{macro}{\tfl@name@i}
% The conversion of the supplied float name to a form suitable for
% using as a title is achieved here.  We strip off the first letter,
% convert to uppercase, then convert the rest to lowercase. Thanks
% to Martin Heller for the method to strip the first letter off.
%    \begin{macrocode}
\def\tfl@gennamea#1#2\end{%
  \expandafter\def\csname tfl@name@\romannumeral\tfl@floatcount%
      \expandafter\endcsname{%
    \MakeUppercase{#1}%
    \MakeLowercase{#2}%
  }%
}%
\def\tfl@genname#1{%
  \tfl@gennamea#1\end%
}%
%    \end{macrocode}
% \end{macro}
% \end{macro}
% \end{macro}
% \begin{macro}{\tfl@list}
% \begin{macro}{\listoffloat-names}
% An auxiliary function is needed to define the
%|\listof\emph{float-name}s| macros.  This is to ensure that
% expansion occurs properly.
%    \begin{macrocode}
\def\tfl@list#1#2#3{%
  \newcommand*{#1}{\listof{#2}{#3}}%
}%
%    \end{macrocode}
% \end{macro}
% \end{macro}
%
% \subsection{User space macros}
%
% \begin{macro}{\trivfloat}
% The only user-facing macro we need is |\trivfloat|. This does the
% work of setting up the new float type. Thanks to Heiko Oberdiek for
% the method to expand arguments correctly.  Notice that a single
% command for the extension would work.  However, the float name does
% need different commands for each float.  So for consistency, both
% have numerals appended.
%    \begin{macrocode}
\newcommand*{\trivfloat}[1]{%
  \advance\tfl@floatcount\@ne%
  \tfl@genext%
  \tfl@genname{#1}%
  \PackageInfo{trivfloat}%
    {Listing all floats of type #1 in \jobname.%
     \csname tfl@ext@\romannumeral\tfl@floatcount\endcsname}%
  \begingroup%
    \edef\x{\endgroup%
      \noexpand\newfloat{#1}{tbp}{%
        \csname tfl@ext@\romannumeral\tfl@floatcount\endcsname%
      }%
      \noexpand\floatname{#1}{%
        \expandafter\noexpand%
        \csname tfl@name@\romannumeral\tfl@floatcount\endcsname%
      }%
      \noexpand\tfl@list{\csname listof#1s\endcsname}{#1}{%
        List of \expandafter\noexpand%
        \csname tfl@name@\romannumeral\tfl@floatcount\endcsname s%
      }%
    }%
  \x%
}%
%    \end{macrocode}
% \end{macro}
%
% \iffalse
%</package>
% \fi
% \PrintChanges
% \PrintIndex
% \Finale
